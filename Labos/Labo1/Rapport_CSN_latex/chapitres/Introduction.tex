\section{Introduction}

Dans le cadre de notre cours IFS (Interface), nous étudions les interactions entre le CPU et les entrées/sorties des systèmes embarqués. Nous utilisons la carte REPTAR de la HEIG-VD composé principalement d'une FPGA Spartan 6 et d'un module DM3730 basé sur un microprocesseur Cortex-A8.\\\par
Ce laboratoire sert de lien entre les notions vues en cours (écriture et lecture des entrée/sorties et création d'interfaces permettant ces lectures et écritures par le CPU) et la mise en pratique de ces notions.\\\par
L'objectif, comme préciser dans la donnée du laboratoire, est de "concevoir une interface pour des I/Os simples,disponible sur la FPGA de la carte REPTAR et de la mettre en œuvre à l'aide d'un programme C". On nous précise d'avance dans la donnée que cela implique deux partie, l'une concernant le développement et l'implémentation matérielle d'une interface(avec la mise en place d'un plan d'adressage et la création du schéma Logisim découlant de notre interface), et l'autre impliquant le développement et l'implémentation d'un programme C permettant de tester et d'exécuter une application utilisant les entrées/sorties de notre interface.\\\par
Il reste à noter le découpage du laboratoire en deux parties, l'une appuyant sur la mise en place d'une interface simple et sur son utilisation, l'autre permettant de mettre en place des fonctionnalités plus avancées et nécessitant l'étude des timings du local bus du DM3730.\\\par
Ce rapport est basé sur le chablon fournit sur Cyberlearn et chaque section excepté l'introduction et la conclusion est séparé en deux partie, appelées "Partie 1" et "Partie 2" qui correspondent aux deux phase du laboratoire (Implémentation simple et ajout de fonctionnalités).

