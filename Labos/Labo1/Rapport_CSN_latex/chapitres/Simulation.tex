\section{Simulation}

\subsection*{Partie 1 :}
\par
Toutes les simulations de cette partie ont été effectuées sur le circuit suivant:\\
\begin{center}
\includegraphics[scale=0.8]{./images/circTest_partie1.png}\\\par
\captionof{figure}{Circuit de simulation}
\end{center}\par
Ci dessous, la simulation de la lecture des switchs sur le circuit:\\
\begin{center}
\includegraphics[scale=0.6]{./images/Test_Part1_Read_Timing_Bus_Reptar.png}\\\par
\captionof{figure}{Simulation de la lecture des switchs}
\end{center}\par
En respectant les timings du bus local reptar, j'ai effectuer la simulation ci-dessus sous logisim. On peut voir que la donnée n'est chargé sur la sortie que lorsque le rd\_en\_i est actif, comme l'on pouvait s'y attendre en ayant étudié les bus reptar et l'interface logisim.\\\par
les deux simulations ci dessous montrent le fonctionnement des écritures sur les sorties : \\
\begin{center}
\includegraphics[scale=0.65]{./images/Test_Part1_WriteAFF_Timing_Bus_Reptar.png}
\captionof{figure}{Simulation de l'ecriture sur l'afficheur 7 segments}
\end{center}
\begin{center}
\includegraphics[scale=0.75]{./images/Test_Part1_WriteLED_Timing_Bus_Reptar.png}\\\par
\captionof{figure}{Simulation de l'écriture sur les leds}
\end{center}\par
On voit que le fonctionnement est en adéquation avec le fonctionnement du timing bus local de la carte REPTAR. Le fonctionnement est le même pour les deux , la valeur des sorties étant mise à jour des qu'une écriture se fait.
\subsection*{Partie 2 :}
Deux simulations différentes ont été fait sur cette partie, la première concernant la relecture des leds, que voici :\\ 
\begin{center}
\includegraphics[scale=0.7]{./images/Test_Part2_Read_Write_Timing_Bus_Reptar.png}\\\par
\captionof{figure}{Simulation de l'écriture et de la relecture sur les leds}
\end{center}\par
Cette simulation est séparé en deux partie. La première en bleu correspond à l'écriture des leds (Comme vu plus haut) et la deuxième partie représente la lecture de ces mêmes leds. On vois que le tout est fonctionnel en respectant les timings du bus local de la carte Reptar.\\\par
La deuxième solution représente une activation de la valeur du flag d'état actif si SW5 à été pesé et un acquittement de ce flag:
\begin{center}
\includegraphics[scale=0.8]{./images/Test_Part2_Flag_Timing_Bus_Reptar.png}\\\par
\captionof{figure}{Simulation de la gestion du flag SW5}
\end{center}\par
La simulation est de nouveau séparée en deux bloc, le premier en bleu représentant l'activation du flag SW5 sur le flanc montant suivant l'activation du switch SW5 (0x10 correspondant au cinquième bit). Le second en vert montre l'acquittement de la part du CPU avec l'écriture à l'adresse du flag d'une valeur. La simulation est donc un succès.
